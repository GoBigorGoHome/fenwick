%! Author = zjsdu
%! Date = 2022/7/7

% Preamble
\documentclass{ctexbeamer}
\title{树状数组}
\author{邹家树}
% Packages
\usepackage{amsmath}
\usepackage{tcolorbox}
\usepackage{annotate-equations}

\newcommand{\lb}{\mathsf{lowbit}}
\newcommand{\popc}{\mathsf{popcount}}
\newcommand{\str}[1]{\texttt{#1}}
\newcommand{\xto}[1]{\xrightarrow{\text{#1}}}
% Document
\begin{document}
\maketitle

\begin{frame}{正整数的二进制表示}
\begin{equation*}
10 =
\eqnmarkbox[blue]{msb}{1}
0
\eqnmarkbox[purple]{lowbit}{1}
\eqnmarkbox[green]{lsb}{0}_2
\end{equation*}

\annotate[yshift=0.5em]{left}{msb}{最高位,第$3$位}
\annotate[yshift=-0.5em]{below}{lsb}{最低位,第$0$位}
\annotate[yshift=0.5em]{right}{lowbit}{最低非零位}

\end{frame}


\begin{frame}{popcount 函数}

对于非负整数$n$,定义函数$\popc(n)$ 为$n$的二进制表示里$1$的个数。

例如$\popc(10) = 2$,$\popc(2^{k}) = 1$,$\popc(0) = 0$。

\end{frame}


\begin{frame}{lowbit函数}
对于正整数 $x$ 定义函数$\lb(x)$为$x$的二进制表示里最低非零位对应的幂。

\begin{block}{}
  例如 $\lb(10) = 2$,$\lb(2^{n}) = 2^{n}$,$\lb(2k+1) = 1$。
\end{block}

\begin{block}{}
  \begin{equation*}
  10 =
  1
  0
  \eqnmarkbox[blue]{lowbit}{1}
  0_2
  \end{equation*}
\end{block}
\annotate[yshift=0.5em]{}{lowbit}{lowbit}

规定 $\lb(0) = 0$。
\end{frame}


\begin{frame}[fragile]{计算 lowbit(x)}

\begin{tcolorbox}
  lowbit(x) = x \& -x
\end{tcolorbox}

  $-x$的补码是 $x$的二进制表示取反再加一。

以 $x = 10$ 为例,$-10$ 的补码可由 $10$的二进制表示经取反和加一两操作得到

\vspace{2em}
\begin{equation*}
  \eqnmarkbox[red]{high1}{\str{10}}
  \eqnmarkbox[blue]{low1}{\str{10}}
  \xto{取反} \str{0101} \xto{加一}
  \eqnmarkbox[red]{high2}{\str{01}}
  \eqnmarkbox[blue]{low2}{\str{10}}
\end{equation*}

\annotatetwo[yshift=-1em]{below}{low1}{low2}{lowbit 以下的位不变}
\annotatetwo[yshift=1em]{above}{high1}{high2}{lowbit 以上的位取反}

\vspace{1em}
\begin{tcolorbox}
  \[\str{.....10000} \xto{取反} \str{*****01111} \xto{加一} \str{*****10000}\]
\end{tcolorbox}


\end{frame}


\begin{frame}{介绍}
对于整数序列$A_1, A_2, \dots, A_n$,定义整数序列$B_1, B_2, \dots, B_n$ 如下

\begin{equation*}
  B_i = A_{i-\lb(i) + 1} + A_{i-\lb(i) + 2} + \dots + A_i,
\end{equation*}

换言之,$B_i$ 是数列 $A$ 里从第 $i$ 项往前的连续 $\lb(i)$ 项之和。

数列 $B$ 可以用来求数列 $A$ 的前缀和,例如

\begin{equation*}
A_1 + \dots + A_{10} = B_{10} + B_{8}
\end{equation*}

\tcbox{$A_1 + \dots + A_i$ 可以表为 $B$ 里 $\popc(i)$ 项之和。}
\end{frame}

\end{document}